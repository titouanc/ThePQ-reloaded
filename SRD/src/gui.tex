Nous avons choisi d'utiliser la SFML (simple and fast multimedia library) dans l'optique de notre projet. 

Cette librairie open-source a été codée en C++ spécialement pour le développement de jeux 2D, et permet l'utilisation de la programmation orientée objet, s'accordant donc avec l'optique de ce travail : apprendre le développement d'applications à des étudiants universitaires. De plus, le site web de la librairie\footnote{http://www.sfml-dev.org/} regorge de documentation et tutoriels.

Une alternative envisagée était SDL, fer de lance du développement de jeux 2D, mais l'utilisation intensive d'OpenGL par la SFML rend celle-ci souvent plus rapide dans son exécution\footnote{http://en.sfml-dev.org/forums/index.php?topic=43.0}.

Enfin, pour ne rien gâcher, la SFML possède une communauté très active, et tout particulièrement la communauté francophone, le développeur étant lui-même Français.