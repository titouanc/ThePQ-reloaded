\subsection{Le championnat}

\begin{figure}[h!]
  \centering
  \includegraphics[width=\textwidth]{figures/ChampionshipUseCase.eps}
  \caption{\label{fig:UC:Championship} Cas d'utilisation: championnat}
\end{figure}

%-----
\subsubsection{Inscription à un championnat (championship entry)}
\paragraph{Précondition}
L'utilisateur n'est pas inscrit à un autre championnat et  le niveau de l'équipe de l'utilisateur est inclus entre les bornes de niveau définies par le championnat.
\paragraph{Cas général} L'utilisateur s'inscrit dans un championnat spécifique.

%-----
\subsubsection{Vue des informations du championnat (championship info)}
	\paragraph{Précondition} L'utilisateur est inscrit à ce championnat.
	\paragraph{Cas général} Le joueur a accès aux informations des équipes concurrentes 
		  (nom et niveau de l'équipe, listing des joueurs,...).

%-----
\subsubsection{Désinscription d'une équipe (unsubscribe)}
	\paragraph{Précondition} L'utilisateur est inscrit, le championnat n'est pas encore 
		  complet et n'a pas encore commencé.
	\paragraph{Cas général} L'utilisateur retire son équipe du championnat.
	\paragraph{Postcondition} L'utilisateur ne participe plus au championnat et n'a plus accès à toutes les informations dudit championnat.

%-----
\subsubsection{Réception de la grille des matches (games schedule)}
	\paragraph{Précondition} L'utilisateur est inscrit au championnat et le nombre 
	      d'équipes est au complet.
	\paragraph{Cas général} Chaque équipe reçoit la grille des matches qui contient la date, l'heure et l'adversaire de chaque match.

%-----
\subsubsection{Jeu des matches (play)}
	\paragraph{Précondition} Le nombre d'équipes est au complet et la grille de matches a été distribuée.
	\paragraph{Cas général} Les utilisateurs disputent un match aller et un match retour l'un  après l'autre.
	\paragraph{Cas exceptionnels} Un des utilisateurs n'est pas présent : l'ordinateur résout le jeu automatiquement.
	\paragraph{Postcondition} Le vainqueur d'un match gagne 3 points au classement général, le perdant ne gagne aucun point, et les 2 équipes gagnent 1 point en cas de match nul. Les aptitudes des joueurs s'améliorent.

%-----
\subsubsection{Réception des résultats (results)}
	\paragraph{Précondition} Tous les matches ont été joués.
	\paragraph{Cas général} L'utilisateur voit le classement général des matches et reçoit éventuellement des prix, de l'expérience et de l'argent.

\subsection{Matches amicaux}
\begin{figure}[h!]
  \centering
  \includegraphics[width=\textwidth]{figures/FriendlyGameUseCase.eps}
  \caption{\label{fig:UC:FriendlyMatch} Cas d'utilisation: matches amicaux}
\end{figure}
\begin{figure}[h!]
  \centering
  \includegraphics[width=\textwidth]{figures/Activity-Match.eps}
  \caption{\label{fig:UC:act-FriendlyMatch} Cas d'utilisation: diagramme d'activité : matches amicaux}
\end{figure}

%-----
\subsubsection{Recherche d'un utilisateur (search for spec. user)}
\paragraph{Cas général} L'utilisateur peut entrer le nom d'un autre utilisateur et lancer sa recherche parmi tous les utilisateurs du jeu, connectés ou non.
\paragraph{Postcondition} Une liste des utilisateurs pouvant correspondre à la recherche est affichée.

\subsubsection{Inviter un utilisateur (invite user)}

\paragraph{Cas général} L'utilisateur envoie une invitation à jouer un match amical à un autre utilisateur.
\paragraph{Postcondition} L'invitation est en attente chez l'autre utilisateur.

\subsubsection{Accepter une invitation (accept invite)}

\paragraph{Précondition} Un autre utilisateur a invité l'utilisateur courant à jouer, et cet utilisateur est en ligne.
\paragraph{Cas général} L'utilisateur accepte cette invitation.

\subsubsection{Refuser une invitation (deny invite)}

\paragraph{Précondition} Un autre utilisateur a invité l’utilisateur courant à jouer et, cet utilisateur est en ligne.

\paragraph{Cas général} L’utilisateur refuse cette invitation.

\paragraph{Postcondition} L’utilisateur est renovyé au menu précedent.

\subsubsection{Jouer une partie (play game)}

\paragraph{Précondition} Les deux utilisateurs ont accepté de jouer ensemble et sont tous les deux connectés.
\paragraph{Cas général} La partie commence.

\subsubsection{Fin de partie}

\paragraph{Précondition} Le jeu est terminé.
\paragraph{Postcondition} Les joueurs des deux utilisateurs progressent.
