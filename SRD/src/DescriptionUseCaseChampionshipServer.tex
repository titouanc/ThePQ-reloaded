
\subsection{Cas d'utilisation du championnat}

\begin{figure}[h]
  \centering
  \includegraphics[width=\textwidth]{figures/UseCaseChampionshipServer.eps}
  \caption{\label{fig:UCS:championshipManagement}Cas d'utilisation: Système - championnat}
\end{figure}

\subsubsection{Recevoir une inscription (receive inscription)}
\label{UCS:receiveInscription}
\paragraph{Relations avec d'autres cas d'utilisation}
Reçoit un message de \ref{UCS:sendInscription}.\\
Inclut \ref{UCS:searchChampionship}.
\paragraph{Pré-conditions}
\begin{list}{\labelitemi}{\leftmargin=1.5em}
\item{Le serveur est allumé.}
\end{list}
\paragraph{Cas général}
Le serveur reçoit une demande d'inscription à un championnat, cette demande venant d'un client (utilisateur).
\paragraph{Post-conditions}
\begin{list}{\labelitemi}{\leftmargin=1.5em}
\item{Une demande d'inscription doit être traitée.}
\end{list}

\subsubsection{Envoyer une requête d'inscription (send inscription request)}
\label{UCS:sendInscription}
\paragraph{Relations avec d'autres cas d'utilisation}
Envoie un message à \ref{UCS:receiveInscription}.
\paragraph{Cas général}
\paragraph{Post-conditions}
\begin{list}{\labelitemi}{\leftmargin=1.5em}
\item{Une requête d'inscription à un championnat est envoyée au serveur.}
\end{list}

\subsubsection{Rechercher un championnat approprié (search adapted championship)}
\label{UCS:searchChampionship}
\paragraph{Relations avec d'autres cas d'utilisation}
Inclus par \ref{UCS:receiveInscription}.\\
Inclut \ref{UCS:createChampionship} et \ref{UCS:addTeam}.
\paragraph{Pré-conditions}
\begin{list}{\labelitemi}{\leftmargin=1.5em}
\item{Une demande d'inscription a été reçue.}
\end{list}
\paragraph{Cas général}
Le serveur effectue une recherche parmi tous les championnats dont le niveau est plus ou moins équivalent à celui de l'équipe de l'utilisateur effectuant la demande.

\subsubsection{Créer un nouveau championnat (create new championship)}
\label{UCS:createChampionship}
\paragraph{Relations avec d'autres cas d'utilisation}
Inclus par \ref{UCS:searchChampionship}.\\
Inclut \ref{UCS:addTeam}.
\paragraph{Pré-conditions}
\begin{list}{\labelitemi}{\leftmargin=1.5em}
\item{Un championnat adéquat a été recherché.}
\item{Aucun championnat adéquat libre (qui n'est pas plein donc). }
\end{list}
\paragraph{Cas général}
Le serveur n'a pas trouvé de championnat correspondant au niveau de l'équipe effectuant la demande d'inscription, il en crée donc un ayant ce niveau.
\paragraph{Post-conditions}
\begin{list}{\labelitemi}{\leftmargin=1.5em}
\item{Un nouveau championnat dont le niveau est mis à celui de l'équipe effectuant la demande d'inscription est créé.}
\end{list}

\subsubsection{Ajouter l'équipe dans le championnat (add user team in championship)}
\label{UCS:addTeam}
\paragraph{Relations avec d'autres cas d'utilisation}
Est inclus par \ref{UCS:searchChampionship} et \ref{UCS:createChampionship}.\\
Inclut \ref{UCS:startChampionship}.
\paragraph{Pré-conditions}
\begin{list}{\labelitemi}{\leftmargin=1.5em}
\item{Un championnat a été trouvé ou créé.}
\end{list}
\paragraph{Cas général}
L'équipe recherchant un championnat adéquat est ajouté au championnat trouvé. 
\paragraph{Post-conditions}
\begin{list}{\labelitemi}{\leftmargin=1.5em}
\item{Le championnat trouvé possède une équipe en plus.}
\end{list}

\subsubsection{Démarrer le championnat (start championship)}
\label{UCS:startChampionship}
\paragraph{Relations avec d'autres cas d'utilisation}
Est inclus par \ref{UCS:addTeam}.\\
Inclut \ref{UCS:createST}.
\paragraph{Pré-conditions}
\begin{list}{\labelitemi}{\leftmargin=1.5em}
\item{L'équipe précédemment ajoutée au championnat a complété ce dernier (le championnat est donc plein).}
\end{list}
\paragraph{Cas général}
L'équipe qui recherchait un championnat adéquat est ajouté à celui-ci. Si cette équipe remplit le dernier emplacement libre de ce championnat, le championnat peut démarrer.
\paragraph{Post-conditions}
\begin{list}{\labelitemi}{\leftmargin=1.5em}
\item{Le championnat est démarré.}
\end{list}

\subsubsection{Créer programme des matches et le classement (create scheduling and starting)}
\label{UCS:createST}
\paragraph{Relations avec d'autres cas d'utilisation}
Inclus par \ref{UCS:startChampionship}.
\paragraph{Pré-conditions}
\begin{list}{\labelitemi}{\leftmargin=1.5em}
\item{Un nouveau championnat démarre.}
\end{list}
\paragraph{Cas général}
Lors du démarrage d'un nouveau championnat, le serveur crée un programme des matches (dates, adversaires) et un classement regroupant les équipes du championnat et leurs points respectifs.
\paragraph{Post-conditions}
\begin{list}{\labelitemi}{\leftmargin=1.5em}
\item{Un programme des matches et un classement est créé pour le championnat.}
\end{list}