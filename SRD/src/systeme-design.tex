Le système fonctionne selon le paradigme \gls{client}-\gls{serveur}.
Lorsqu'un client essaie de se connecter au serveur, celui-ci crée un fil d'exécution 
spécifique au client. Le client se connecte et entre dans la partie gestion de 
l'application. Lorsqu'un match démarre pour le client, le serveur crée un 
nouveau processus pour le match séparé en trois fils d'exécution, un pour 
chaque client jouant le match et un pour la gestion du match. A la fin du 
match, ce processus est tué et les résultats du match sont renvoyés au 
processus principal.
Le système sera programmé en C++, avec l'aide de la bibliothèque standard et 
d'une bibliothèque pour la partie graphique. Ce dernier point doit encore 
être discuté avec les clients et l'équipe technique.

\subsection{Composants}
\begin{figure}[h!]
  \centering
  \includegraphics[width=\textwidth]{figures/Components.eps}
  \caption{\label{fig:Components} Composants}
\end{figure}
Les différents use cases présentés antérieurement seront implémentés grâces aux composants décrits à la figure \ref{fig:Components}

\subsection{Envoi des messages}
Pour l'envoi de données sur le réseau et la persistance des données, nous 
avons envisagé une seule et même solution : la sérialisation d'objets JSON (illustrée à la figure \ref{fig:Class:JSON})
(JavaScript Object Notation). A l'aide d'un outil de sérialisation/désérialisation JSON de notre 
cru, nous allons transformer les objets à envoyer ou stocker en objets JSON. 
Ces objets JSON possèderont une méthode \textit{toString} pour en faire une 
chaine de caractères stockable dans un fichier texte ou envoyable sur le 
réseau.

\begin{figure}[h!]
  \centering
  \includegraphics[width=\textwidth]{figures/JSON-Class.eps}
  \caption{\label{fig:Class:JSON} Classes: outil de sérialisation JSON}
\end{figure}

\subsection{Phase de jeu}
\begin{figure}[h!]
  \centering
  \includegraphics[width=0.7\textwidth]{figures/Game-Sequence.eps}
  \caption{\label{fig:Sequence:Game} Séquence : le jeu}
\end{figure}
La figure \ref{fig:Sequence:Game} illustre les messages envoyés entre le \gls{client} et le \gls{serveur} 
lors d'une phase de jeu. 
Le serveur possède un processus dédié pour chaque match en cours. C'est de 
ce processus qu'il est question dans cette section.
Le client indique d'abord qu'il est présent, ce qui permet au serveur de 
savoir si une résolution automatique doit avoir lieu. 
Ensuite le serveur envoie les informations du match au client. Dans le cas 
d'une résolution automatique, le serveur envoie les résultats du match.
Quand le client a chargé le match, il indique qu'il est prêt et le jeu commence.
Lors de chaque tour, le client envoie les coups joués par l'utilisateur et le 
serveur renvoie aux deux clients les mises à jour de l'état de la partie.
Un message spécial du serveur lors de la mise à jour indique que la partie
est terminée et le serveur envoie alors les résultats du match dans un autre 
message. Quand le client les a reçues, il le signifie au serveur qui ferme 
alors la connexion.

\subsection{Le gestionnaire de connexion}
\begin{figure}[h!]
  \centering
  \includegraphics[width=\textwidth]{figures/network-class.eps}
  \caption{\label{fig:Class:Network} Classes: Le gestionnaire de connexion}
\end{figure}

Les connexions sont ici gérées dans le \emph{BaseConnectionManager}, qui instancie deux threads : un s'occupant des messages entrant, l'autre des sortant. Ces messages arrivent en JSON via un file descriptor, via une file (suivant leur entrée ou sortie) afin d'arriver dans le bon ordre.

Le \emph{ConnectionManager} ajoute à la précédente classe la possibilité d'accepter de nouveaux file descriptors.

Le \emph{SubConnectionManager} peut voler un des file descriptors le temps d'un match, fonctionnalité utilisée par le serveur lors des matches.



\subsubsection{Le gestionnaire de match}
\begin{figure}[h!]
  \centering
  \includegraphics[width=0.7\textwidth]{figures/Game-Class_3_MatchManager.eps}
  \caption{\label{fig:Class:MatchManager} Classes: le gestionnaire de match}
\end{figure}

Le gestionnaire de match comprend les quatre balles, les deux équipes et les coups du 
tour actuel. Il gère tout ce qui concerne le match. Il s'occupe de l'échange 
des messages préliminaires et des résultats. Lors de chaque tour : côté 
client, il récolte les coups pour les deux équipes et puis les envoie au 
\gls{serveur}, côté serveur, il résout les coups simultanément et renvoie le résultat 
au \gls{client}.

\subsubsection{Les objets déplaçables du match}
\begin{figure}[h!]
  \centering
  \includegraphics[width=\textwidth]{figures/Game-Class_1_Moveables.eps}
  \caption{\label{fig:Class:GameMoveables} Classes: les objets déplaçables du match}
\end{figure}

Tous les objets déplaçables dans un match (joueurs et balles) proviennent d'une 
même interface \textbf{Moveable}. Les \textbf{Moveable} possèdent une vitesse 
et une position. La vitesse correspond au nombre de cases parcourables en un tour 
et la position est un couple $(x,y)$ dans un repère cartésien. Les directions sont des vecteurs constants représentant un déplacement d'1 pied dans une des directions possibles.

\subsubsection{Les actions des joueurs}
\begin{figure}[h!]
  \centering
  \includegraphics[width=0.7\textwidth]{figures/Game-Class_2_Stroke.eps}
  \caption{\label{fig:Class:Stroke} Classes: les actions des joueurs}
\end{figure}

Les actions sont au nombre de quatre : déplacement, coup de batte, sort et lancer 
du Souafle. On peut vérifier si une action est logique. Chaque action correspond 
à un coup qui est envoyé au \gls{serveur} et qui peut être décomposé pour permettre la 
"simultanéité" des événements.

\subsection{Phase de gestion}
\begin{sidewaysfigure}
  \centering
  \includegraphics[width=\textwidth]{figures/Gestion-Class.eps}
  \caption{\label{fig:Class:Management} Classes: la gestion de l'équipe}
\end{sidewaysfigure}

Un \gls{manager} contrôle une \gls{equipe}, c'est-à-dire des \gls{membre}s 
(\gls{joueur}s et 
\gls{coach}), des \gls{installation}s dans son \gls{stade} et des 
\gls{sponsor}s. Chaque équipe a également des 
fonds, une réputation, un nombre matchs gagné et un nombre de matchs total.
Les sponsors rapportent de l'argent à chaque match et nécessitent un minimum 
de réputation de l'équipe pour y être liée. Les installations coûtent 
de l'argent à chaque intervalle de temps et rapporte de l'argent lors de 
chaque match.
Les membres ont un salaire, qui est déduit à chaque intervalle de temps des 
fonds de l'équipe et un prix d'achat sur le marché des transferts.
Les joueurs possèdent une barre de vie qui diminue lors de blessures en match 
et une barre de mana qui diminue en lançant des sorts. Ils ont également des 
aptitudes qui varient selon leur position de jeu favorite.
Les coachs possèdent des aptitudes qui permettent d'améliorer les aptitudes 
des joueurs.

\subsubsection{Les championnats}
\begin{figure}[h!]
  \centering
  \includegraphics[width=0.7\textwidth]{figures/Championship-Class.eps}
  \caption{\label{fig:Class:Championship} Classes: les championnats}
\end{figure}
Le championnat se joue avec un système de "poule" de huit équipes. Chaque 
\gls{selection} joue un 
match aller-retour contre toutes les autres équipes. Une victoire rapporte 
trois 
points à l'équipe victorieuse, un match nul un point pour les deux équipes et une défaite 
ne rapporte aucun point. Au terme du championnat, nous obtenons un classement général
en fonction des points de toutes les équipes (cfr. classe Ranking). Si on se trouve face
à un cas d'égalité, on départage les équipes en fonction du goal average. 


Une fois que le championnat commence, le serveur envoie la grille de matches à toutes 
les équipes. Cette grille de matches est gérée par la classe Schedule qui connaît tous les
matches à jouer et les matches joués, y compris leur score.




