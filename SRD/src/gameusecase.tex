\subsection{Phase de jeu}
Dans ce mode de jeu, l'utilisateur déplace ses joueurs sur un terrain où se
déplacent les joueurs d'une autre équipe et des balles. L'utilisateur joue au
tour par tour jusqu'à ce que le match soit fini. L'utilisateur doit voir une 
représentation du terrain de jeu, doit pouvoir y sélectionner un joueur et lui 
ordonner des actions comme un déplacement, un lancement de balle ou un lancement 
de sort. Lors de chaque tour, la vue du terrain est rafraîchie pour refléter 
l'état courant du match, les joueurs choisissent chacun simultanément les coups
à jouer pour chacun de leurs joueurs, dans une limite de temps impartie, et 
chaque joueur ne pouvant se déplacer que de la distance qu'il peut parcourir en
un tour. Quand le joueur a terminé de jouer, ou que le temps est dépassé, les
coups à jouer sont envoyés au serveur, qui va calculer les déplacements, les 
points gagnés et les collisions qui interviennent durant ce tour.

\begin{figure}[h!]
    \centering
    \includegraphics[width=\textwidth]{figures/Game-UseCase.eps}
    \caption{\label{fig:UC:game} Cas d'utilisation: match de Quidditch}
\end{figure}

\subsubsection{Sélection d'un joueur (select player)}
    \label{UC:selectPlayerOnPitch}
    \paragraph{Précondition} L'utilisateur est connecté au match et le temps de jeu pour ce tour n'est pas dépassé.
    \paragraph{Cas général} L'utilisateur clique sur un joueur. L'ensemble des cases accessibles en ligne droite depuis la position du joueur est mis en évidence. Des icônes apparaissent selon les particularités du joueur: frapper la balle, attraper le vif d'or, lancer un sort d'amélioration, etc\footnote{L'ensemble des coups jouables sera déterminé plus loin dans le développement du jeu, à partir des constats sur l'équilibrage du jeu}...
    \paragraph{Cas exceptionnels} Le \gls{joueur} n'appartient pas à la \gls{selection} de l'\gls{utilisateur}.
    \paragraph{Postcondition} le joueur est sélectionné. 

\subsubsection{Sélection d'une direction (choose direction)}
    \label{UC:selectDirection}
    \paragraph{Précondition} Un joueur est sélectionné ou une action du joueur sélectionné requiert une direction.
    \paragraph{Cas général} L'utilisateur clique sur une case accessible en ligne droite.
    \paragraph{Cas exceptionnel} La case de destination n'est pas accessible à cause des règles de l'action à effectuer.
    \paragraph{Postcondition} L'action est ajoutée à la liste des coups de l'utilisateur pour ce tour; une icône est ajoutée au groupe d'icônes des actions de l'utilisateur pour ce tour.

\subsubsection{Sélection d'une destination (select destination)}
    \label{UC:selectDestination}
    \paragraph{Précondition} Un joueur est sélectionné.
    \paragraph{Cas général} L'utilisateur clique sur une des cases accessibles en ligne droite \textit{(mises en évidence au cas d'utilisation \ref{UC:selectPlayerOnPitch})} selon le cas d'utilisation \ref{UC:selectDirection}.
    \paragraph{Postcondition} Si le joueur continue un mouvement, ce déplacement est ajouté au mouvement du joueur. Sinon, le joueur commence un mouvement, et une icône est ajoutée au groupe d'icône des actions du tour.

\subsubsection{Lancement d'une balle (throw ball)}
    \paragraph{Précondition} Le joueur sélectionné est un \gls{poursuiveur} et il a le \gls{souaffle}.
    \paragraph{Cas général} Continue selon le cas d'utilisation \ref{UC:selectDirection}

\subsubsection{Utiliser la batte (use batt)}
    \paragraph{Précondition} Le joueur sélectionné est un \gls{batteur}.
    \paragraph{Cas général} Le joueur essaye de taper dans un \gls{cognard}. Lorsque le mouvement sera effectivement joué, si le joueur est au même moment sur la même case que la balle, celle-ci partira dans la direction spécifiée, sinon le joueur tape dans le vent. Continue selon le cas d'utilisation \ref{UC:selectDirection} .

\subsubsection{Lancer un sort (cast spell)}
    \label{UC:castSpell}
    \paragraph{Précondition} Le sort est utilisable à ce moment (les conditions peuvent varier selon le sort et l'état du jeu).
    \paragraph{Cas général} L'utilisateur clique sur un des sorts du joueur. Si le sort nécessite une direction, continue selon le cas d'utilisation \ref{UC:selectDirection}. Sinon, une icône s'ajoute (comme au cas d'utilisation précité).
    \paragraph{Postcondition} L'action est ajoutée à la liste des coups pour ce tour.

\subsubsection{Annuler un coup (cancel action)}
    \paragraph{Précondition} L'utilisateur a au moins joué un coup durant ce \gls{tour}.
    \paragraph{Cas général} L'utilisateur effectue un clic droit sur une des icônes de coups joués.
    \paragraph{Postcondition} Le coup est retiré de la liste des coups pour ce tour.

\subsubsection{Attraper le vif d'or (catch golden snitch)}
    \paragraph{Précondition} L'attrapeur est assez proche du vif d'or pour le voir et l'attraper (selon ses caractéristiques).
    \paragraph{Cas général} Se déroule comme au cas d'utilisation \ref{UC:castSpell}.
    \paragraph{Postcondition} Lorsque le tour est joué: si le sort réussit le match s'arrête et l'\gls{equipe} du joueur remporte 150 points.

\subsubsection{Terminer son tour volontairement (end turn)}
    \paragraph{Précondition} Le temps de jeu pour ce tour n'est pas dépassé.
    \paragraph{Cas général} L'utilisateur clique sur le bouton "terminer le tour". Continue selon le cas d'utilisation \ref{UC:endTurn}.

\subsubsection{Fin de temps de jeu}
    \paragraph{Précondition} Le temps de jeu imparti pour ce tour est dépassé.
    \paragraph{Cas général} Continue selon le cas d'utilisation \ref{UC:endTurn}.

\subsubsection{Terminer un tour}
    \label{UC:endTurn}
    \paragraph{Cas général} La liste des coups pour l'utilisateur est envoyée au serveur et les icônes qui les représentent disparaissent. Si les deux utilisateurs terminent leurs tours avant la fin du temps imparti, le tour est exécuté directement.
    \paragraph{Postcondition} Tous les coups sont confirmés et l'utilisateur ne peut plus les modifier. L'utilisateur attend que le tour de son adversaire soit lui aussi terminé. Les coups sont joués sur le serveur et l'utilisateur voit le nouvel état de jeu.

\subsection{Cas d'utilisation indirects}
Ces cas d'utilisation décrivent comment évolue une partie de jeu si certaines actions des joueurs ont lieu, mais sur lesquelles l'utilisateur n'a pas influence directe.

\subsubsection{Attraper le souaffle}
    \paragraph{Précondition} Un \gls{poursuiveur} et le \gls{souaffle} sont sur une même case à un instant donné.
    \paragraph{Cas général} Selon les caractéristiques du joueur, la probabilité que la balle soit attrapée est plus ou moins élevée. Un tir aléatoire détermine si le joueur l'attrape ou non.
    \paragraph{Postcondition} Si le tir aléatoire a déterminé que le joueur attrape la balle, il l'a en main et peut la lancer.

\subsubsection{Marquer un but}
    \paragraph{Précondition} Un \gls{poursuiveur} a lancé le souaffle et celui-ci passe par une case goal de l'\gls{equipe} adverse. 
    \paragraph{Cas général} La balle est placée une case derrière le goal, à l'arrêt.
    \paragraph{Postcondition} L'équipe qui ne possède pas la case goal à travers laquelle le souaffle est passé reçoit 10 points.

\subsubsection{Sortir la balle du terrain}
    \paragraph{Précondition} Une balle dépasse la limite du terrain.
    \paragraph{Postcondition} La balle est arrêtée à la dernière position occupée sur le terrain.
